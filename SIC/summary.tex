\documentclass[a4paper,12pt,twoside]{article}

\usepackage{ucs}
\usepackage[utf8x]{inputenc}
\usepackage[brazil]{babel}
\usepackage[T1]{fontenc}
\usepackage{graphicx}
\usepackage{caption}
\usepackage{subcaption}
\usepackage[margin=2.5cm]{geometry}
\usepackage{times}

\usepackage{mathtools}
\pagenumbering{gobble}
\title{ 
Desenvolvimento de uma interface gráfica para a ferramenta \textit{LTSE} \\ ~ \\
\normalsize Levindo Gabriel Taschetto Neto \\ Orientador Lucio Mauro Duarte\\
\small Instituto de Informática - Universidade Federal do Rio Grande do Sul
}

\date{}

\begin{document}

\maketitle
\vspace{-1.7cm}


% Estrutura:
% Introdução:
%  Desenvolvimento:
%  -> Metodologia utilizados
%  -> Resultados obtidos até o momento
%  Conclusão: encerrar

% Blocos:
% Bloco #1 - Contexto: Descrição do contexto no qual o trabalho está inserido
%            para quem não é da área, algo como: “A extensão para o LTSE contribui de tal
%            maneira para os usuários...".
% Bloco #2 - Problema: Descrição do problema que está sendo abordado. No caso,
%            o de facilitar o uso da plataforma LTSE, e também pensei em colocar o problema
%            que o próprio LTSE resolve.
% Bloco #3 - Metodologia/solução/resultados: Descrição do trabalho que de fato
%            foi feito (tentar não focar tanto na implementação em si, mas em sua estrutura
%            e maneira que está organizada).
% Bloco #4 - Conclusão/trabalhos futuros: Descrição do benefício esperado
%            ("esse trabalho ajudará usuários a ter um melhor uso da plataforma", etc) e os
%            planos para trabalhos futuros (a ideia é mostrar nessa parte que essa linha de
%            pesquisa não é um “dead-end”, e que pode gerar mais desdobramentos).


%------------------------------------------- Introdução [Contexto] ------------------------------------------
No projeto \textit{VeriTeS} estão sendo desenvolvidos trabalhos na área de validação, verificação e testes de sistemas computacionais com uso de técnicas e ferramentas com foco na análise de modelos de comportamento de software.
O projeto tem como objetivo o desenvolvimento de diretivas para o desenvolvimento de software baseado em modelos.
Dentro desse escopo, a ferramenta \textit{Labelled Transition Systems Extractor (LTSE)} é utilizada para a obtenção de modelos a partir de códigos existentes, habilitando a análise de sistemas já em execução.

%------------------------------------------------ Problema----------------------------------------------------
Todavia, a ferramenta possuía problemas de \textit{input} de dados, em função de dificuldades em ter parâmetros de entradas fornecidos para a mesma.
Ademais, a visualização dos dados gerados era um problema, uma vez que não se tinha uma interface gráfica para apresentar os resultados obtidos depois da execução da aplicação.
Um dos principais problemas que a extensão desenvolvida resolve é o de facilitação do uso do \textit{LTSE} para todos os utilizadores da ferramenta. Isso se deve ao fato de que a plataforma, em suas versões mais antigas, disponibilizava apenas a inserção de comandos por meio de linha de comando.
Dessa maneira com a qual a inserção de informações era realizada, a ferramenta era mais suscetível a gerar resultados incorretos do ponto de vista do usuário.
Além disso, a usabilidade da plataforma era afetada diretamente, uma vez que usuários precisavam saber determinados comandos e formatos de parâmetro de terminal para executar a ferramenta com as configurações desejadas.

%--------------------------------------- Metodologia/Solução/Resultados --------------------------------------

A solução encontrada para facilitar a utilização da ferramenta \textit{LTSE} foi a da criação de uma extensão \textit{GUI (graphical user interface)} para a mesma.
A extensão foi desenvolvida na linguagem de programação Java, a qual é a mesma na qual a ferramenta \textit{LTSE} foi desenvolvida.
Conceitos de orientação a objetos foram fortemente utilizados ao longo do desenvolvimento, de forma que a aplicação tivesse: (i) níveis de abstração bem definidos, (ii) flexibilidade no fluxo de dados que são trocados entre usuário e o \textit{backend} do \textit{LTSE}, e (iii) fosse fácil de ser estendida e integrada a outras ferramentas.
A \textit{GUI} encontra-se disponível de maneira \textit{open source} e pode ser facilmente integrada em projetos do mesmo cunho teórico e estrutural da ferramenta \textit{LTSE}, podendo assim, auxiliar também no trabalho de outros pesquisadores.

%----------------------------------------- Conclusão/Trabalhos futuros ----------------------------------------

Trabalhos futuros incluem explorar outras propriedades de modelagem de software dentro da extensão desenvolvida.
Ademais, testes unitários e de integração serão criados para a \textit{GUI}, fazendo com que a mesma se torne ainda mais robusta e confiável.
Trabalhos já estão em andamento para o melhoramento do design da extensão criada, com a utilização de componentes adaptados da biblioteca de interface Material-UI.
Todo trabalho desenvolvido até o momento se concentra em auxiliar usuários a poder gerar e avaliar modelos de software de maneira simples, rápida e eficaz.

\end{document}
